\documentclass[12pt,a4paper]{amsart}

% Usual preamble

\usepackage[T1]{fontenc}
\usepackage[utf8]{inputenc}
\usepackage[british]{babel}
\usepackage{mathtools}
\usepackage{amsthm}
\usepackage{libertine}
\usepackage[libertine]{newtxmath}
\usepackage[mathscr]{euscript}
\usepackage{enumitem}
\usepackage{tikz-cd}
\usetikzlibrary{decorations.markings}
\usepackage{float}
\usepackage[
  backend=biber,
  style=alphabetic,
  maxnames=10,
  maxalphanames=10]{biblatex}
\addbibresource{refs.bib}
\usepackage{hyperref}
\urlstyle{same}
\usepackage[noabbrev]{cleveref}

\theoremstyle{plain}
\newtheorem{thm}{Theorem}
\newtheorem*{thm*}{Theorem}
\newtheorem{lm}[thm]{Lemma}
\newtheorem{prop}[thm]{Proposition}
\newtheorem{cor}[thm]{Corollary}
\newtheorem{fact}[thm]{Fact}
\newtheorem{q}[thm]{Question}
\theoremstyle{definition}
\newtheorem{defn}[thm]{Definition}
\newtheorem{nota}[thm]{Notation}
\newtheorem{exmp}[thm]{Example}
\newtheorem{xca}[thm]{Exercise}
\theoremstyle{remark}
\newtheorem{rem}[thm]{Remark}
\Crefname{thm}{Theorem}{Theorems}
\Crefname{lm}{Lemma}{Lemmas}
\Crefname{prop}{Proposition}{Propositions}
\Crefname{cor}{Corollary}{Corollaries}
\Crefname{fact}{Fact}{Facts}
\Crefname{q}{Question}{Questions}
\Crefname{defn}{Definition}{Definitions}
\Crefname{nota}{Notation}{Notations}
\Crefname{exmp}{Example}{Examples}
\Crefname{xca}{Exercise}{Exercises}
\Crefname{rem}{Remark}{Remarks}

% Specific preamble

\title[Brown Representability]{Brown Representability}
\author[Pedro N\'{u}\~{n}ez]{Pedro N\'{u}\~{n}ez}\thanks{{\itshape Email:}~\href{mailto:pedro.nunez@math.uni-freiburg.de}{pedro.nunez@math.uni-freiburg.de}. \\ \indent {\itshape Homepage:}~\href{https://home.mathematik.uni-freiburg.de/nunez/?l=en}{https://home.mathematik.uni-freiburg.de/nunez}}
\date{\today}

\setcounter{tocdepth}{1}
\sloppy
\makeatletter
\hypersetup{
  pdfauthor={\authors},
  pdftitle={\@title},
  colorlinks,
  linkcolor=[rgb]{0.2,0.2,0.6},
  citecolor=[rgb]{0.2,0.2,0.6},
  urlcolor=[rgb]{0.2,0.2,0.6}}
\makeatother

\begin{document}

\maketitle

\tableofcontents

\section{Why cohomology?}

We would like be able to distinguish between different topological spaces.
For example, it is useful to know that we live in a sphere rather than in a (hollow) donut.
In order to ensure that two topological spaces are different---not isomorphic as topological spaces---we would need to show that there does not exist a homeomorphism between them.
But this is a very difficult problem in general.
For example, how do we show that there is no homeomorphism between the $2$-sphere and the donut?
There are plenty of bijections between the two spaces, so checking directly that each of these bijections is not continuous is not a viable option.

Instead, we look for \emph{invariants} of topological spaces, i.e., quantities or objects that are the same among isomorphic topological spaces.
If two topological spaces have different invariants, then they cannot be homeomorphic.
For example, the number of points that we need to remove from a topological space before it becomes contractible is an invariant which allows us to distinguish the $2$-sphere from the donut.
It is easy to see that this invariant differs for these two spaces, but it can be difficult to compute this invariant for arbitrary topological spaces.
Instead, it is more convenient to look at invariants which are perhaps a bit harder to compute in such easy examples, but which can be computed in a more systematic way for arbitrary topological spaces.
Cohomology is one such invariant.

\begin{rem}
  Homology would also be such an invariant, but cohomology has more interesting structure to it---the \emph{cup product}---and it is closer to homotopy theory in a sense made precise by Brown representability.
  There is an analogous statement for homology, but it is more complicated to obtain \cite[\S 4.F]{hat02}.
\end{rem}

\section{(Singular) cohomology}

A \emph{(singular) $n$-simplex} on a topological space $X$ is a continuous image of the standard $n$-simplex $\Delta^{n} \subseteq \mathbb{R}^{n+1}$.
We want to keep track of the order of the vertices and we may replace the standard $n$-simplex by any other $n$-dimensional simplex of the form
\[ [v_{0}, \ldots, v_{n}] := \left\{ \sum_{i = 0}^{n} t_{i}v_{i} \,\middle|\, t_{i} \in [0,1] \text{ for all } i \in \{0,\ldots, n\} \text{ and } \sum_{i = 0}^{n} t_{i} = 1 \right\}, \]
where the $v_{0}, \ldots, v_{n} \in \mathbb{R}^{n+1}$ form an (ordered) basis.
We denote by $C_{n}(X)$ the free abelian group generated by $n$-simplices in $X$.
We have \emph{boundary maps} as follows:

\begin{align*}
  d_{n-1} \colon C_{n}(X) & \longrightarrow C_{n-1}(X) \\
  (\sigma \colon [v_{0},\ldots,v_{n}] \to X) & \longmapsto \sum_{i = 0}^{n} (-1)^{i}\sigma|_{[v_{0}, \ldots, \hat{v}_{i}, \ldots, v_{n}]}.
\end{align*}

Some combinatorics \cite[Lemma 2.1]{hat02} show that $d_{n} \circ d_{n+1} = 0$ for all $n \in \mathbb{N}$, so $(C_{\bullet}(X),d_{\bullet})$ is a \emph{chain complex}.
We can dualize it by applying the functor $\operatorname{Hom}(-,\mathbb{Z})$ to obtain a cochain complex $(C^{\bullet}(X),d^{\bullet})$.
Explicitly, the \emph{coboundary map} $d^{n} \colon C^{n}(X) \to C^{n+1}(X)$ is given by

\begin{equation}\label{eqn:coboundary}
  d^{n}(\varphi)(\sigma) = \sum_{i = 0}^{n+1} (-1)^{i}\varphi(\sigma|_{[v_{0}, \ldots, \hat{v}_{i}, \ldots, v_{n+1}]}).
\end{equation}

\begin{defn}[Cohomology groups]
  For each $n \in \mathbb{N}$, we define the $n$-th (singular) \emph{cohomology group} of $X$ as
  \[ H^{n}(X) := H^{n}(C^{\bullet}(X)) = \ker(d^{n})/\operatorname{im}(d^{n-1}). \]
\end{defn}

\begin{exmp}
  Let $\varphi \colon C_{0}(X) \to \mathbb{Z}$ be a homomorphism.
  Then $\varphi \in \ker(d^{0})$ if and only if $\varphi(x) = \varphi(y)$ for every pair of points $x, y \in X$ in the same path-connected component, where we think of the points $x$ and $y$ as $0$-simplices.
  So $H^{0}(X)$ are the functions from the set of path-connected components of $X$ to $\mathbb{Z}$, i.e., a product of copies of $\mathbb{Z}$ indexed by the path-connected components of $X$.
\end{exmp}

For each $n \in \mathbb{N}$, this construction yields a \emph{contravariant functor} $H^{n}(-)$ from topological spaces with continuous maps to abelian groups.
We can extend this functor to the category of pairs $(X,A)$ consisting of a topological space $X$ and a subspace $A \subseteq X$ as follows.
We have a natural inclusion $C_{n}(A) \subseteq C_{n}(X)$ and $d_{n-1}(C_{n}(A)) \subseteq C_{n-1}(A)$, so setting $C_{n}(X,A) := C_{n}(X)/C_{n}(A)$ we obtain a new chain complex $C_{\bullet}(X,A)$ with boundary maps induced by the ones on $C_{\bullet}(X)$.
We again define $C^{\bullet}(X,A)$ as the dual cochain complex $\operatorname{Hom}(C_{\bullet}(X,A),\mathbb{Z})$ and set
\[ H^{n}(X,A) := H^{n}(C^{\bullet}(X,A)). \]

\begin{rem}
  Sice $C_{n}(A) \subseteq C_{n}(X)$ is the free abelian subgroup spanned by a subset of the generators of $C_{n}(X)$, the quotient $C_{n}(X,A)$ is again a free abelian group.
  This implies that the short exact sequence
  \[ 0 \to C_{n}(A) \to C_{n}(X) \to C_{n}(X,A) \to 0 \]
  remains exact after applying $\operatorname{Hom}(-,\mathbb{Z})$.
  Therefore we can think of $C^{n}(X,A)$ as the subset of functions in $C^{n}(X)$ which vanish on all $n$-simplices with image contained in $A$.
\end{rem}

\begin{defn}[Reduced cohomology]
  Let now $(X,x_{0})$ be a pointed space.
  For each $n \in \mathbb{N}$ we define the \emph{reduced cohomology} of $X$ as
  \[ \tilde{H}^{n}(X) = H^{n}(X,\{x_{0}\}) = \ker(H^{n}(X) \to H^{n}(\{x_{0}\})). \]
\end{defn}

\begin{rem}
  We can recover the unreduced cohomology of pairs from the reduced one.
  See \emph{Relation between reduced and unreduced cohomology} on \href{https://ncatlab.org/nlab/show/generalized+\%28Eilenberg-Steenrod\%29+cohomology\#RelationBetweenReducedAndUnreduced}{nLab's page for generalized (Eilenberg--Steenrod) cohomology}.
\end{rem}

Reduced cohomology gives us then a family of functors $\{ \tilde{H}^{n}(-) \}_{n \in \mathbb{N}}$ from pointed topological spaces to abelian groups.
In \Cref{sec:cohomologies} we will discuss some of its properties, but before we want to discuss a particularly nice kind of topological spaces to which we want to restrict our attention.

\section{CW complexes}

\begin{defn}[CW complex]
  For $n \in \mathbb{N}$, an \emph{$n$-cell} is a topological space homeomorhpic to the closed unit ball $D^{n} \subseteq \mathbb{R}^{n}$.
  A \emph{CW complex} is a topological space $X$ constructed inductively as follows:
  \begin{enumerate}
    \item The $0$-skeleton $X^{0}$ is a disjoint union of $0$-cells.
    \item The $m$-skeleton $X^{m}$ is obtained from $X^{m-1}$ by attaching a collection of $m$-cells to it along their boundaries.
      That is, there exists a set $A$ and a collection of maps $\{\varphi_{\alpha} \colon S^{m-1} \to X^{m-1}\}_{\alpha \in A}$ such that $X^{m}$ fits into the following pushout diagram:
      \begin{center}
        \begin{tikzcd}
          \sqcup_{\alpha \in A} S^{m-1}_{\alpha} \arrow[hook]{r} \arrow{d}{\sqcup_{\alpha \in A}} & \sqcup_{\alpha \in A} D^{m}_{\alpha} \arrow{d} \\
          X^{m-1} \arrow{r} & X^{m} 
        \end{tikzcd}
      \end{center}
    \item We set $X = \cup_{m \in \mathbb{N}} X^{m}$ and endow it with the weak topology, i.e., $U \subseteq X$ is open if and only if $U \cap X^{m}$ is open in $X^{m}$ for all $m \in \mathbb{N}$.
  \end{enumerate}
\end{defn}

Many familiar topological spaces can be described as CW complexes:

\begin{exmp}
  The $2$-sphere can be described as CW complex in many ways, for example:
  \begin{itemize}
    \item One single $0$-cell and one single $2$-cell.
    \item One single $0$-cell, one single $1$-cell and two $2$-cells.
  \end{itemize}
\end{exmp}

\begin{rem}
  Some reasons to restrict our attention to CW complexes:

  \begin{itemize}
    \item A practical reason:~We may compute the cohomology of a CW complex via cellular cohomology \cite[p.~202]{hat02}, which is usually much easier to compute than singular cohomology but yields the same result.
      For example, in order to compute cellular cohomology of a CW complex without cells on adjacent degrees---e.g., higher dimensional spheres or complex projective spaces---, we only need to count the number of cells on each degree.
    \item A conceptual reason:~CW complexes form a \href{https://ncatlab.org/nlab/show/convenient+category+of+topological+spaces}{convenient category of topological spaces}.
      Moreover, every topological space is weakly homotopic to a CW comlpex \cite[Proposition 4.13]{hat02}, and a map between simply connected CW complexes inducing isomorphisms on homology groups is automatically a homotopy equivalence \cite[Corollary 4.33]{hat02}.
  \end{itemize}
\end{rem}

\section{Cohomology theories}\label{sec:cohomologies}

We have seen how to attach some algebraic invariant---singular cohomology---to topological spaces, and we've agreed on restricting our attention to CW complexes.
This invariant turns out to be very useful, so we would like to axiomatize its most useful features in order to find similar invariants.

We start by analyzing what we currently have.
We are considering pointed CW complexes and continuous pointed maps between them.
This data forms a \emph{category}, which we will simply denote as $\mathbf{Top}$.
This means that we can compose maps---\emph{morphisms}---in an associative manner and that each CW complex---each \emph{object} in the category---has an identity morphism to itself, satisfying the usual identity axioms.
We denote by $\mathbf{Top}^{\mathrm{op}}$ the opposite category, which has the same objects and all arrows inverted, i.e., a morphism $Y \to X$ for every continuous map $X \to Y$.
Another example of category is the category $\mathbf{Ab}$ of abelian groups and group homomorphisms.
For each $n \in \mathbb{N}$ we have a \emph{functor}
\[ \tilde{H}^{n} \colon \mathbf{Top}^{\mathrm{op}} \to \mathbf{Ab}, \]
which associates to each object (resp.~morphism) in $\mathbf{Top}$ an object (resp.~morphism) in $\mathbf{Ab}$ preserving composition and identity morphisms.
We can also regard this as a family of functors indexed by $n \in \mathbb{Z}$; the value on negative degrees is then always zero by definition.
A key feature of cohomology, which allows us to relate the $\tilde{H}^{n}$ for different $n$, is the \emph{suspension isomorphism}:

\begin{defn}[Suspension]
  Let $(X,\{ x_{0} \})$ be a pointed topological space.
  We define its \emph{(reduced) suspension} as the quotient
  \[ \Sigma X := (X \times [0,1])/(X \times \{ 0 \} \cup X \times \{ 1 \} \cup \{ x_{0} \} \times I) \]
  equipped with the base-point $[(x_{0},0)]$.
\end{defn}

For each $n \in \mathbb{Z}$ and each CW complex $X$ we get a \emph{natural} isomorphism $\tilde{H}^{n+1}(\Sigma X) \cong \tilde{H}^{n}(X)$, where naturality means that these isomorphisms are compatible with morphisms in $\mathbf{Top}$.
This key feature will essentially allow us to restrict our attention to a functor $\tilde{H}^{n}$ for a single $n \in \mathbb{Z}$.

The next fundamental feature of cohomology is its \emph{homotopy invariance}.
Suppose $f, g \colon X \to Y$ are homotopic maps, i.e., there exists a continuous map $F \colon X \times [0,1] \to Y$ such that $F|_{X \times \{ 0 \}} = f$ and $F|_{X \times \{ 1 \}} = g$.
Then $\tilde{H}^{n}(f) = \tilde{H}^{n}(g)$ for all $n \in \mathbb{Z}$, i.e., $f$ and $g$ induce the same homomorphisms in cohomology.
Homotopy of continuous maps defines an equivalence relation compatible with comopsition, so we may consider the \emph{homotopy category} $\mathbf{hTop}$ whose objects are CW complexes and whose morphisms are homotopy classes of continuous maps between them.
Homotopy invariance of $\tilde{H}^{n}$ translates then into saying that this functor induces a well-defined functor
\[ \tilde{H}^{n} \colon \mathbf{hTop} \to \mathbf{Ab}. \]

The category $\mathbf{hTop}$ has now some nice extra properties, e.g., we can form ``direct sums'' of two spaces $(X,\{x_{0}\})$ and $(Y,\{y_{0}\})$ by taking their \emph{wedge sum}
\[ X \vee Y := X \cup_{x_{0} \sim y_{0}} Y. \]
Singular cohomology is then compatible with wedge sums in the sense that the canonical morphisms
\[ \tilde{H}^{n}( \vee_{i\in I} X_{i} ) \to \prod_{i \in I} \tilde{H}^{n}(X_{i}) \]
are isomorphisms for all $n \in \mathbb{Z}$.
Since finite direct products of abelian groups are the same as finite direct sums, we can think of these wedge sum isomorhpisms as ``additivity'' of the cohomology functors.
Finally, we can consider some kind of ``short exact sequence'' in $\mathbf{hTop}$.
Namely, if $A \subseteq X$ is the inclusion of a subcomplex, then we can consider the maps
\[ A \hookrightarrow X \twoheadrightarrow X/A. \]
The last key property that we need is that every such ``short exact sequence'' induces a short exact sequence
\[ 0 \to \tilde{H}^{n}(X/A) \to \tilde{H}^{n}(X) \to \tilde{H}^{n}(A) \to 0 \]
for all $n \in \mathbb{Z}$, i.e., the functors $\tilde{H}^{n}$ are ``exact''.

We use these properties to axiomatize cohomology:

\begin{defn}[Cohomology theorey]
  A \emph{(reduced) cohomology theory} is a family of functors
  \[ h^{n} \colon \mathbf{hTop} \to \mathbf{Ab} \]
  for $n \in \mathbb{Z}$ satisfying the following axioms:
  \begin{enumerate}[label=(\roman*)]
    \item Suspension isomorhpism.
    \item Additivity.
    \item Exactness.
  \end{enumerate}
\end{defn}

\begin{rem}
  Other common sets of axioms include the long exact sequence assocaited to an inclusion $A \subseteq X$.
  This long exact sequence can be deduced from exactness and the suspension isomorhpism, using that $X/A$ is homotopy equivalent to the result of attaching the cone over $A$ to $X$ \cite[Theorem VII.1.6]{bre93} and that we can obtain $\Sigma A$ by collapsing $X$ after attaching the cone over $A$ to $X$ \cite[Corollary VII.5.5]{bre93}.
\end{rem}

\begin{rem}
  One of the fundamental results about cohomology theories is that if we further assume that $h^{n}(S^{0}) = 0$ for all $n \neq 0$, then this cohomology theory is isomorphic to singular cohomology \cite[Theorem 4.59]{hat02}.
\end{rem}

\section{Spectra}

Given any CW complex $K$, we may consider the corresponding \emph{representable functor}
\[ [-,K] \colon \mathbf{hTop}^{\mathrm{op}} \to \mathbf{Set} \]
which sends a CW complex $X$ to the set of homotopy classes of maps from $X$ to $K$.
More generally, a functor is called \emph{representable} if it is naturally isomorphic to one such functor.
Representable functors have many nice properties:~They are compatible with direct sums and similar constructions (transform colimits in $\mathbf{hTop}$ into limits in $\mathbf{Set}$) and they determine the object $K$ itself (Yoneda lemma).
Wouldn't it be nice if cohomology functors were representable?

The additivity of cohomology functors insinuates that we may hope that they are representable.
But why would the set $[X,K]$ carry any sort of natural abelian group structure to begin with?
This is where \emph{spectra} come into play.

The basic idea is the following.
Attached to a CW complex $X$, we may consider its \emph{loopspace}
\[ \Omega K := \{ \gamma \colon S^{1} \to K \mid \gamma \text{ continuous} \} \]
equipped with the compact-open topology.
We can comopse loops up to homotopy as one does when defining the fundamental group $\pi_{1}(K)$, so given $f,g \in [X, \Omega K]$ we may add them as
\[ (f + g)(x) := f(x) \cdot g(x), \]
where $\cdot$ denotes composition of loops.
This makes $[X,\Omega K]$ into a group, but not necessarily an abelian group.
However, this turns out to be the case if we iterate this construction one more time and we consider $[X, \Omega^{2} K]$ instead.
This can be deduced from the \href{https://en.wikipedia.org/wiki/Eckmann\%E2\%80\%93Hilton_argument}{Eckmann--Hilton principle} as in the case of higher homotopy groups.

Let's say then that we want our cohomology functors to be represented by loopspaces.
It would then be nice if the suspension isomorphisms between the cohomology functors came from some relation among the different loopspaces representing the cohomology on each degree.
To see what this relation should be, we use another nice property of the loopspace construction:~It is \emph{adjoint} to the suspension, i.e., there are natural isomorphisms
\[ [\Sigma X, Y] \cong [X, \Omega Y]. \]
In particular, if we consider the family of spaces $\{ K_{n} \}_{n \in \mathbb{N}}$ such that $K_{n} \cong \Omega K_{n+1}$ for all $n \in \mathbb{N}$, then we would have isomorphisms
\[ [ X, K_{n}] \cong [ X, \Omega K_{n+1}] = [ \Sigma X,  K_{n+1}], \]
resembling the suspension isomorhpism that we would have if $h^{n}(X)$ were isomorhpic to $[X,K_{n}]$ for all $n \in \mathbb{N}$.
Moreover, now each set
\[ [X, K_{n}] \cong [X, \Omega^{2} K_{n+2}] \]
would carry a natural abelian group structure.

Since each $K_{n}$ determines $K_{n-1}$, we can extend such a family of spaces to all negative integers.
This leads to the following definition:

\begin{defn}[Spectrum]
  An \emph{$\Omega$-spectrum} is a family of CW complexes\footnote{A result of Milnor ensures that the loopspace of a CW comlpex is homotopy equivalent to a CW complex.} $\{K_{n}\}_{n \in \mathbb{N}}$ together with isomorphisms $K_{n} \cong \Omega K_{n+1}$ for all $n \in \mathbb{N}$.
\end{defn}

We then have the following:

\begin{thm}[{\cite[Theorem 4.58]{hat02}}]
  Let $\{K_{n}\}_{n \in \mathbb{N}}$ be an $\Omega$-spectrum.
  Then the functors
  \[ h^{n}(X) := [X,K_{n}] \]
  for $n \in \mathbb{Z}$ define a cohomology theory.
\end{thm}

\section{Brown Representability theorem}

Therefore, every $\Omega$-spectrum induces a cohomology theory.
\emph{Brown representability} states that the converse is true as well:

\begin{thm}[{\cite[Theorem 4E.1]{hat02}}]
  Every (reduced) cohomology theory is induced by an $\Omega$-spectrum $\{ K_{n} \}_{n \in \mathbb{N}}$ via
  \[ h^{n}(X) = [X, K_{n}]. \]
\end{thm}

\printbibliography
\vfill

\end{document}
