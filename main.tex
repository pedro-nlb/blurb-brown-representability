\documentclass[12pt,a4paper]{amsart}

% Usual preamble

\usepackage[T1]{fontenc}
\usepackage[utf8]{inputenc}
\usepackage[british]{babel}
\usepackage{mathtools}
\usepackage{amsthm}
\usepackage{libertine}
\usepackage[libertine]{newtxmath}
\usepackage[mathscr]{euscript}
\usepackage{enumitem}
\usepackage{tikz-cd}
\usetikzlibrary{decorations.markings}
\usepackage{float}
\usepackage[
  backend=biber,
  style=alphabetic,
  maxnames=10,
  maxalphanames=10]{biblatex}
\addbibresource{refs.bib}
\usepackage{hyperref}
\urlstyle{same}
\usepackage[noabbrev]{cleveref}

\theoremstyle{plain}
\newtheorem{thm}{Theorem}
\newtheorem*{thm*}{Theorem}
\newtheorem{lm}[thm]{Lemma}
\newtheorem{prop}[thm]{Proposition}
\newtheorem{cor}[thm]{Corollary}
\newtheorem{fact}[thm]{Fact}
\newtheorem{q}[thm]{Question}
\theoremstyle{definition}
\newtheorem{defn}[thm]{Definition}
\newtheorem{nota}[thm]{Notation}
\newtheorem{exmp}[thm]{Example}
\newtheorem{xca}[thm]{Exercise}
\theoremstyle{remark}
\newtheorem{rem}[thm]{Remark}
\Crefname{thm}{Theorem}{Theorems}
\Crefname{lm}{Lemma}{Lemmas}
\Crefname{prop}{Proposition}{Propositions}
\Crefname{cor}{Corollary}{Corollaries}
\Crefname{fact}{Fact}{Facts}
\Crefname{q}{Question}{Questions}
\Crefname{defn}{Definition}{Definitions}
\Crefname{nota}{Notation}{Notations}
\Crefname{exmp}{Example}{Examples}
\Crefname{xca}{Exercise}{Exercises}
\Crefname{rem}{Remark}{Remarks}

% Specific preamble

\title[Brown Representability]{Brown Representability}
\author[Pedro N\'{u}\~{n}ez]{Pedro N\'{u}\~{n}ez}\thanks{{\itshape Email:}~\href{mailto:pedro.nunez@math.uni-freiburg.de}{pedro.nunez@math.uni-freiburg.de}. \\ \indent {\itshape Homepage:}~\href{https://home.mathematik.uni-freiburg.de/nunez/?l=en}{https://home.mathematik.uni-freiburg.de/nunez}}
\date{\today}

\setcounter{tocdepth}{1}
\sloppy
\makeatletter
\hypersetup{
  pdfauthor={\authors},
  pdftitle={\@title},
  colorlinks,
  linkcolor=[rgb]{0.2,0.2,0.6},
  citecolor=[rgb]{0.2,0.2,0.6},
  urlcolor=[rgb]{0.2,0.2,0.6}}
\makeatother

\begin{document}

\maketitle

\tableofcontents

\section{Why cohomology?}

We would like be able to distinguish between different topological spaces.
For example, it is useful to know that we live in a sphere rather than in a (hollow) donut.
In order to ensure that two topological spaces are different---not isomorphic as topological spaces---we would need to show that there does not exist a homeomorphism between them.
But this is a very difficult problem in general.
For example, how do we show that there is no homeomorphism between the $2$-sphere and the donut?
There are plenty of bijections between the two spaces, so checking directly that each of these bijections is not continuous is not a viable option.

Instead, we look for \emph{invariants} of topological spaces, i.e., quantities or objects that are the same among isomorphic topological spaces.
If two topological spaces have different invariants, then they cannot be homeomorphic.
For example, the number of points that we need to remove from a topological space before it becomes contractible is an invariant which allows us to distinguish the $2$-sphere from the donut.
It is easy to see that this invariant differs for these two spaces, but it can be difficult to compute this invariant for arbitrary topological spaces.
Instead, it is more convenient to look at invariants which are perhaps a bit harder to compute in such easy examples, but which can be computed in a more systematic way for arbitrary topological spaces.
Cohomology is one such invariant.

\section{Cell complexes and cellular cohomology}

\begin{defn}[Cell complex]
  For $n \in \mathbb{N}$, an \emph{$n$-cell} is a topological space homeomorhpic to the closed unit ball $D^{n} \subseteq \mathbb{R}^{n}$.
  A \emph{cell complex} is a topological space $X$ constructed inductively as follows:
  \begin{enumerate}
    \item The $0$-skeleton $X^{0}$ is a disjoint union of $0$-cells.
    \item The $m$-skeleton $X^{m}$ is obtained from $X^{m-1}$ by attaching a collection of $m$-cells to it along their boundaries.
      That is, there exists a set $A$ and a collection of maps $\{\varphi_{\alpha} \colon S^{m-1} \to X^{m-1}\}_{\alpha \in A}$ such that $X^{m}$ fits into the following pushout diagram:
      \begin{center}
        \begin{tikzcd}
          \sqcup_{\alpha \in A} S^{m-1}_{\alpha} \arrow[hook]{r} \arrow{d}{\sqcup_{\alpha \in A}} & \sqcup_{\alpha \in A} D^{m}_{\alpha} \arrow{d} \\
          X^{m-1} \arrow{r} & X^{m} 
        \end{tikzcd}
      \end{center}
    \item We set $X = \cup_{m \in \mathbb{N}} X^{m}$ and endow it with the weak topology, i.e., $U \subseteq X$ is open if and only if $U \cap X^{m}$ is open in $X^{m}$ for all $m \in \mathbb{N}$.
  \end{enumerate}
\end{defn}

Many familiar topological spaces can be described as cell complexes:

\begin{exmp}
  The $2$-sphere can be described as cell complex in many ways, for example:
  \begin{itemize}
    \item One single $0$-cell and one single $2$-cell.
    \item One single $0$-cell, one single $1$-cell and two $2$-cells.
  \end{itemize}
\end{exmp}

We can use a cell complex description of a topological space to compute its (cellular) cohomology:

\begin{defn}[Cellular cohomology]
  ...
\end{defn}

\section{Cohomology theories}

\section{Spectra}

\section{Brown Representability theorem}

\printbibliography
\vfill

\end{document}
