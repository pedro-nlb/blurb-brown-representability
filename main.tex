\documentclass[12pt,a4paper]{amsart}

% Usual preamble

\usepackage[T1]{fontenc}
\usepackage[utf8]{inputenc}
\usepackage[british]{babel}
\usepackage{mathtools}
\usepackage{amsthm}
\usepackage{libertine}
\usepackage[libertine]{newtxmath}
\usepackage[mathscr]{euscript}
\usepackage{enumitem}
\usepackage{tikz-cd}
\usetikzlibrary{decorations.markings}
\usepackage{float}
\usepackage[
  backend=biber,
  style=alphabetic,
  maxnames=10,
  maxalphanames=10]{biblatex}
\addbibresource{refs.bib}
\usepackage{hyperref}
\urlstyle{same}
\usepackage[noabbrev]{cleveref}

\theoremstyle{plain}
\newtheorem{thm}{Theorem}
\newtheorem*{thm*}{Theorem}
\newtheorem{lm}[thm]{Lemma}
\newtheorem{prop}[thm]{Proposition}
\newtheorem{cor}[thm]{Corollary}
\newtheorem{fact}[thm]{Fact}
\newtheorem{q}[thm]{Question}
\theoremstyle{definition}
\newtheorem{defn}[thm]{Definition}
\newtheorem{nota}[thm]{Notation}
\newtheorem{exmp}[thm]{Example}
\newtheorem{xca}[thm]{Exercise}
\theoremstyle{remark}
\newtheorem{rem}[thm]{Remark}
\Crefname{thm}{Theorem}{Theorems}
\Crefname{lm}{Lemma}{Lemmas}
\Crefname{prop}{Proposition}{Propositions}
\Crefname{cor}{Corollary}{Corollaries}
\Crefname{fact}{Fact}{Facts}
\Crefname{q}{Question}{Questions}
\Crefname{defn}{Definition}{Definitions}
\Crefname{nota}{Notation}{Notations}
\Crefname{exmp}{Example}{Examples}
\Crefname{xca}{Exercise}{Exercises}
\Crefname{rem}{Remark}{Remarks}

% Specific preamble

\title[Brown Representability]{Brown Representability}
\author[Pedro N\'{u}\~{n}ez]{Pedro N\'{u}\~{n}ez}\thanks{{\itshape Email:}~\href{mailto:pedro.nunez@math.uni-freiburg.de}{pedro.nunez@math.uni-freiburg.de}. \\ \indent {\itshape Homepage:}~\href{https://home.mathematik.uni-freiburg.de/nunez/?l=en}{https://home.mathematik.uni-freiburg.de/nunez}}
\date{\today}

\setcounter{tocdepth}{1}
\sloppy
\makeatletter
\hypersetup{
  pdfauthor={\authors},
  pdftitle={\@title},
  colorlinks,
  linkcolor=[rgb]{0.2,0.2,0.6},
  citecolor=[rgb]{0.2,0.2,0.6},
  urlcolor=[rgb]{0.2,0.2,0.6}}
\makeatother

\begin{document}

\maketitle

\tableofcontents

\section{Why cohomology?}

We would like be able to distinguish between different topological spaces.
For example, it is useful to know that we live in a sphere rather than in a (hollow) donut.
In order to ensure that two topological spaces are different---not isomorphic as topological spaces---we would need to show that there does not exist a homeomorphism between them.
But this is a very difficult problem in general.
For example, how do we show that there is no homeomorphism between the $2$-sphere and the donut?
There are plenty of bijections between the two spaces, so checking directly that each of these bijections is not continuous is not a viable option.

Instead, we look for \emph{invariants} of topological spaces, i.e., quantities or objects that are the same among isomorphic topological spaces.
If two topological spaces have different invariants, then they cannot be homeomorphic.
For example, the number of points that we need to remove from a topological space before it becomes contractible is an invariant which allows us to distinguish the $2$-sphere from the donut.
It is easy to see that this invariant differs for these two spaces, but it can be difficult to compute this invariant for arbitrary topological spaces.
Instead, it is more convenient to look at invariants which are perhaps a bit harder to compute in such easy examples, but which can be computed in a more systematic way for arbitrary topological spaces.
Cohomology is one such invariant.

\begin{rem}
  Homology would also be such an invariant, but cohomology has more interesting structure to it---the \emph{cup product}---and it is closer to homotopy theory in a sense made precise by Brown representability.
  There is an analogous statement for homology, but it is more complicated to obtain \cite[\S 4.F]{hat02}.
\end{rem}

\section{Cell complexes and cellular cohomology}

\subsection{(Singular) cohomology}

A \emph{(singular) $n$-simplex} on a topological space $X$ is a continuous image of the standard $n$-simplex $\Delta^{n} \subseteq \mathbb{R}^{n+1}$.
We want to keep track of the order of the vertices and we may replace the standard $n$-simplex by any other $n$-dimensional simplex of the form
\[ [v_{0}, \ldots, v_{n}] := \left\{ \sum_{i = 0}^{n} t_{i}v_{i} \,\middle|\, t_{i} \in [0,1] \text{ for all } i \in \{0,\ldots, n\} \text{ and } \sum_{i = 0}^{n} t_{i} = 1 \right\}, \]
where the $v_{0}, \ldots, v_{n} \in \mathbb{R}^{n+1}$ form an (ordered) basis.
We denote by $C_{n}(X)$ the free abelian group generated by $n$-simplices in $X$.
We have \emph{boundary maps} as follows:

\begin{align*}
  d_{n-1} \colon C_{n}(X) & \longrightarrow C_{n-1}(X) \\
  (\sigma \colon [v_{0},\ldots,v_{n}] \to X) & \longmapsto \sum_{i = 0}^{n} (-1)^{i}\sigma|_{[v_{0}, \ldots, \hat{v}_{i}, \ldots, v_{n}]}.
\end{align*}

Some combinatorics \cite[Lemma 2.1]{hat02} show that $d_{n} \circ d_{n+1} = 0$ for all $n \in \mathbb{N}$, so $(C_{\bullet}(X),d_{\bullet})$ is a \emph{chain complex}.
We can dualize it by applying the functor $\operatorname{Hom}(-,\mathbb{Z})$ to obtain a cochain complex $(C^{\bullet}(X),d^{\bullet})$.
Explicitly, the \emph{coboundary map} $d^{n} \colon C^{n}(X) \to C^{n+1}(X)$ is given by

\begin{equation}\label{eqn:coboundary}
  d^{n}(\varphi)(\sigma) = \sum_{i = 0}^{n+1} (-1)^{i}\varphi(\sigma|_{[v_{0}, \ldots, \hat{v}_{i}, \ldots, v_{n+1}]}).
\end{equation}

\begin{defn}[Cohomology groups]
  For each $n \in \mathbb{N}$, we define the $n$-th (singular) \emph{cohomology group} of $X$ as
  \[ H^{n}(X) := H^{n}(C^{\bullet}(X)) = \ker(d^{n})/\operatorname{im}(d^{n-1}). \]
\end{defn}

\begin{exmp}
  Let $\varphi \colon C_{0}(X) \to \mathbb{Z}$ be a homomorphism.
  Then $\varphi \in \ker(d^{0})$ if and only if $\varphi(x) = \varphi(y)$ for every pair of points $x, y \in X$ in the same path-connected component, where we think of the points $x$ and $y$ as $0$-simplices.
  So $H^{0}(X)$ are the functions from the set of path-connected components of $X$ to $\mathbb{Z}$, i.e., a product of copies of $\mathbb{Z}$ indexed by the path-connected components of $X$.
\end{exmp}

For each $n \in \mathbb{N}$, this construction yields a \emph{contravariant functor} $H^{n}(-)$ from topological spaces with continuous maps to abelian groups.
We can extend this functor to the category of pairs $(X,A)$ consisting of a topological space $X$ and a subspace $A \subseteq X$ as follows.
We have a natural inclusion $C_{n}(A) \subseteq C_{n}(X)$ and $d_{n-1}(C_{n}(A)) \subseteq C_{n-1}(A)$, so setting $C_{n}(X,A) := C_{n}(X)/C_{n}(A)$ we obtain a new chain complex $C_{\bullet}(X,A)$ with boundary maps induced by the ones on $C_{\bullet}(X)$.
We again define $C^{\bullet}(X,A)$ as the dual cochain complex $\operatorname{Hom}(C_{\bullet}(X,A),\mathbb{Z})$ and set
\[ H^{n}(X,A) := H^{n}(C^{\bullet}(X,A)). \]
We will discuss the properties of this functor in \Cref{sec:cohomologies}.

\begin{rem}
  Sice $C_{n}(A) \subseteq C_{n}(X)$ is the free abelian subgroup spanned by a subset of the generators of $C_{n}(X)$, the quotient $C_{n}(X,A)$ is again a free abelian group.
  This implies that the short exact sequence
  \[ 0 \to C_{n}(A) \to C_{n}(X) \to C_{n}(X,A) \to 0 \]
  remains exact after applying $\operatorname{Hom}(-,\mathbb{Z})$.
  Therefore we can think of $C^{n}(X,A)$ as the subset of functions in $C^{n}(X)$ which vanish on all $n$-simplices with image contained in $A$.
\end{rem}

\subsection{Cell complexes}

\begin{defn}[Cell complex]
  For $n \in \mathbb{N}$, an \emph{$n$-cell} is a topological space homeomorhpic to the closed unit ball $D^{n} \subseteq \mathbb{R}^{n}$.
  A \emph{cell complex} is a topological space $X$ constructed inductively as follows:
  \begin{enumerate}
    \item The $0$-skeleton $X^{0}$ is a disjoint union of $0$-cells.
    \item The $m$-skeleton $X^{m}$ is obtained from $X^{m-1}$ by attaching a collection of $m$-cells to it along their boundaries.
      That is, there exists a set $A$ and a collection of maps $\{\varphi_{\alpha} \colon S^{m-1} \to X^{m-1}\}_{\alpha \in A}$ such that $X^{m}$ fits into the following pushout diagram:
      \begin{center}
        \begin{tikzcd}
          \sqcup_{\alpha \in A} S^{m-1}_{\alpha} \arrow[hook]{r} \arrow{d}{\sqcup_{\alpha \in A}} & \sqcup_{\alpha \in A} D^{m}_{\alpha} \arrow{d} \\
          X^{m-1} \arrow{r} & X^{m} 
        \end{tikzcd}
      \end{center}
    \item We set $X = \cup_{m \in \mathbb{N}} X^{m}$ and endow it with the weak topology, i.e., $U \subseteq X$ is open if and only if $U \cap X^{m}$ is open in $X^{m}$ for all $m \in \mathbb{N}$.
  \end{enumerate}
\end{defn}

Many familiar topological spaces can be described as cell complexes:

\begin{exmp}
  The $2$-sphere can be described as cell complex in many ways, for example:
  \begin{itemize}
    \item One single $0$-cell and one single $2$-cell.
    \item One single $0$-cell, one single $1$-cell and two $2$-cells.
  \end{itemize}
\end{exmp}

\subsection{Cellular cohomology}

Singular cohomology is hard to compute directly by hand, but for a cell complex we can compute it as follows:

\begin{defn}[Cellular cohomology]
  ...
\end{defn}

\section{Cohomology theories}\label{sec:cohomologies}

\section{Spectra}

\section{Brown Representability theorem}

\printbibliography
\vfill

\end{document}
